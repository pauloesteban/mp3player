\documentclass[12pt,a4paper]{article}
\usepackage[latin1]{inputenc}
\usepackage[spanish]{babel}
\usepackage{amsmath}
\usepackage{amsfonts}
\usepackage{amssymb}
\usepackage{graphicx}
\usepackage{pdfpages}
\usepackage[left=2cm,right=2cm,top=2cm,bottom=2cm]{geometry}
\author{Brito Orlando, Huac�n Linda, Rosales Joshua, Salazar Jes�s}
\begin{document}
\includepdf{CARATULA}
\title{\textbf{INFORME 3}}
\maketitle
El presente informe fue realizado con la finalidad de dar a conocer los cambios implementados en la aplicaci�n a partir de mejoras implementadas conforme se avanza en el desarrollo de la aplicaci�n, a m�s de dar a conocer las nuevas implementaciones realizadas al reproductor Mp3 con base en las planificaciones descritas en el backlog del producto, en este caso los backlog's de las diferentes interfaces de pantalla.
\center{\large{\textbf{NUEVAS IMPLEMENTACIONES}}}
\begin{figure}[h!]
\begin{minipage}{0.5\textwidth}
\centering \includegraphics[width=0.75\textwidth]{imagen.png}     
\end{minipage}
\hfill\begin{minipage}{0.5\textwidth}
\textbf{Pantalla en reproducci�n:} Se realizaron modificaciones en el fondo de la pantalla, con respecto a la pantalla donde se muestran las barras en movimiento de acuerdo al ritmo de la m�sica, en este caso en el fondo estar� la portada del �lbum al que pertenezca la canci�n en ejecuci�n.
Ademas se implementaron cambios en la barra de duracion de la canci�n pasando de la barra horizontal a una barra redonda alrededor del bot�n de play, finalmente tambi�n se resaltaron los botones t�picos del reproductor d�ndoles un rol central en el interfaz.
\end{minipage}
\end{figure}
\pagebreak
\newpage
\includepdf{SPRINT}
\end{document}