\documentclass[12pt,a4paper]{article}
\usepackage[latin1]{inputenc}
\usepackage[spanish]{babel}
\usepackage{amsmath}
\usepackage{amsfonts}
\usepackage{amssymb}
\usepackage{wrapfig}
\usepackage{graphicx}
\usepackage{pdfpages}
\usepackage[left=2cm,right=2cm,top=2cm,bottom=2cm]{geometry}
\setlength{\parindent}{0cm}
\date{}
\author{}
\begin{document}
\includepdf[page=1]{CARATULA}
\begin{center}
\section{INTRODUCCI�N.}
\end{center}
Este documento describe la implementaci�n de la metodolog�a \textbf{scrum}, para el desarrollo de la aplicaci�n de software Reproductor Mp3.
Incluyendo la descripci�n del proceso de este proyecto, mediante avances semanales del desarrollo del mismo, as� como las responsabilidades y compromisos de los participantes en el proyecto.
\\
\section{Objetivo.}
Realizar un avance de la interfaz gr�fica de usuario del reproductor Mp3 totalmente funcional.
\section{Fundamentaci�n.}
Las principales razones del uso de un ciclo de desarrollo iterativo e incremental de tipo scrum para la ejecuci�n de este proyecto son:
\begin{itemize}
\setlength{\parindent}{0cm}
\item Entregas frecuentes y continuas al docente (cliente) de las distintas interfaces terminadas, de forma que pueda disponer de una funcionalidad b�sica en un tiempo m�nimo y a partir de ah� un incremento y mejora continua del Reproductor Mp3.
\item Previsible inestabilidad de requisitos.
\begin{itemize}
\item Es posible que el sistema incorpore m�s funcionalidades de las inicialmente identificadas.
\item Es posible que durante el desarrollo del proyecto se altere el orden en el que se desean recibir las interfaces gr�ficas de usuario terminadas.
\end{itemize}
\end{itemize}
\section{Valores De Trabajo.}
Los valores que deben ser practicados por todos los miembros involucrados en el desarrollo y que hacen posible que la metodolog�a Scrum tenga �xito son:
\begin{itemize}
\item Autonom�a del equipo
\item Respeto en el equipo
\item	Responsabilidad y auto-disciplina
\item Foco en la tarea
\item	Informaci�n transparencia y visibilidad.
\end{itemize}
\newpage
\section{Integrantes Y Roles En El Equipo.}
\begin{table}[htbp]
\centering
\begin{tabular}{|l|l|}
\hline
\multicolumn{2}{|c|}{Roles De Los Integrantes Seg�n Scrum} \\ \hline
Integrante & Rol \\
\hline \hline
 Brito Orlando & Team \\ \hline
 Huac�n Linda & Master \\ \hline
 Rosales Joshua & Manager \\ \hline
 Salazar Jes�s & Team \\ \hline
\end{tabular}
\end{table}
\center{\large{\textbf{PRODUCT MANAGER}}}
\begin{figure}[h!]
\begin{minipage}{0.5\textwidth}
\centering \includegraphics[width=0.75\textwidth]{../../Joshua.jpg}   
\end{minipage}
\hfill\begin{minipage}{0.5\textwidth}
\textbf{Joshua Rosales:} La posici�n del product manager fue inicialmente concebido para gestionar la complejidad de las l�neas de producto, as� como para asegurar que dichos productos sean rentables. Los Jefes de Producto pueden proceder de diferentes carreras profesionales, porque sus capacidades principales incluyen trabajar con clientes y comprender los problemas que el producto est� destinado a resolver.
\end{minipage}
\end{figure}
\center{\large{\textbf{SCRUM MASTER}}}
\begin{figure}[h!]
\begin{minipage}{0.5\textwidth}
\centering \includegraphics[width=0.75\textwidth]{../../20161030_125131.jpg} 
\end{minipage}
\hfill\begin{minipage}{0.5\textwidth}
\textbf{Linda Huac�n:} El Scrum es facilitado por un ScrumMaster, cuyo trabajo primario es eliminar los obst�culos que impiden que el equipo alcance el objetivo del sprint. El ScrumMaster no es el l�der del equipo (porque ellos se auto-organizan), sino que act�a como una protecci�n entre el equipo y cualquier influencia que le distraiga. El ScrumMaster se asegura de que el proceso Scrum se utiliza como es debido. El ScrumMaster es el que hace que las reglas se cumplan.
\end{minipage}
\end{figure}
\newpage
\center{\large{\textbf{SCRUM TEAM}}}
\begin{figure}[h!]
\begin{minipage}{0.5\textwidth}
\centering \includegraphics[width=0.75\textwidth]{../../Orlando.jpg} 
\end{minipage}
\hfill\begin{minipage}{0.5\textwidth}
\textbf{Orlando Brito:} Pertenece al grupo de programadores con los conocimientos t�cnicos necesarios para desarrollar el proyecto de manera conjunta llevando a cabo las especificaciones a las que se comprometen al inicio de cada sprint. Principalmente su rol dentro del team estar� enfocado en la programaci�n del c�digo de toda la aplicaci�n.
\end{minipage}
\end{figure}
\begin{figure}[h!]
\begin{minipage}{0.5\textwidth}
\centering \includegraphics[width=0.75\textwidth]{../../240346_206568909381871_1752317_o.jpg} 
\end{minipage}
\hfill\begin{minipage}{0.5\textwidth}
\textbf{Jes�s Salazar:} Pertenece al grupo de dise�adores con los conocimientos t�cnicos necesarios para desarrollar el proyecto de manera conjunta llevando a cabo las especificaciones a las que se comprometen al inicio de cada sprint. Principalmente su rol dentro del team estar� enfocado en el dise�o del entorno gr�fico de toda la aplicaci�n.
\end{minipage}
\end{figure}
\newpage
\includepdf[pages=1-3]{tablas}
\newpage
\includepdf[page=1]{sprint}
\newpage
\includepdf[page=1-3]{Capturas1}
\end{document}