\documentclass[12pt,a4paper]{article}
\usepackage[latin1]{inputenc}
\usepackage[spanish]{babel}
\usepackage{amsmath}
\usepackage{amsfonts}
\usepackage{amssymb}
\usepackage{pdfpages}
\usepackage{graphicx}
\usepackage[left=2cm,right=2cm,top=2cm,bottom=2cm]{geometry}
\author{Brito Orlando, Huac�n Linda, Rosales Joshua, Salazar Jes�s}
\begin{document}
\includepdf{Caratula}
\title{\textbf{INFORME 2}}
\maketitle
El presente informe fue realizado con la finalidad de dar a conocer los cambios implementados en la aplicaci�n a ra�z de las correcciones realizadas por nuestro cliente (Docente), a m�s de dar a conocer las nuevas implementaciones realizadas al reproductor Mp3 con base en las planificaciones descritas en el backlog del producto, en este caso los backlog's de las diferentes interfaces de pantalla.
\center{\large{\textbf{NUEVAS IMPLEMENTACIONES}}}

\begin{figure}[h!]
\begin{minipage}{0.5\textwidth}
\centering \includegraphics[width=0.75\textwidth]{../../Downloads/15049638_1848410708706880_1423299285_n.png}    
\end{minipage}
\hfill\begin{minipage}{0.5\textwidth}
\textbf{Pantalla principal:} Se realizaron modificaciones en el fondo de la pantalla, pasando de un fondo con im�genes a un fondo con un color plano, as� mismo se eliminaron ciertos botones irrelevantes en el funcionamiento de la aplicaci�n, esto con el fin de evitar posibles confusiones por parte del usuario.
\end{minipage}
\end{figure}

\begin{figure}[h!]
\begin{minipage}{0.5\textwidth}
\centering \includegraphics[width=0.75\textwidth]{../../Downloads/15046223_1848410692040215_168907512_n.png}     
\end{minipage}
\hfill\begin{minipage}{0.5\textwidth}
\textbf{Interfaz Canciones:} Se realizaron modificaciones en el fondo de la pantalla, adem�s se realiz� el intercambio en la ubicaci�n de los botones de b�squeda por nombre y b�squeda por voz. Finalmente se a�adieron las opciones de playlist o lista de reproducci�n y la opci�n de lista de espera.
\end{minipage}
\end{figure}

\begin{figure}[h!]
\begin{minipage}{0.5\textwidth}
\centering \includegraphics[width=0.75\textwidth]{../../Downloads/15032566_1848410682040216_923906413_n.png}      
\end{minipage}
\hfill\begin{minipage}{0.5\textwidth}
\textbf{Interfaz Canciones: Opci�n lista de reproducci�n.-} Se a�adi� la opci�n de playlist o lista de reproducci�n, ya que es algo bastante utilizado por los usuarios de reproductores Mp3 que les permite hacer una selecci�n de todas las canciones que ellos elijan.
\end{minipage}
\end{figure}

\begin{figure}[h!]
\begin{minipage}{0.5\textwidth}
\centering \includegraphics[width=0.75\textwidth]{../../Downloads/15032566_1848410682040216_923906413_n.png}      
\end{minipage}
\hfill\begin{minipage}{0.5\textwidth}
\textbf{Interfaz Canciones: Opci�n lista de espera.-} Se a�adi� la opci�n lista de espera, ya que ofrece a los usuarios la posibilidad de hacer una lista de reproducci�n desechable, es decir el usuario puede elegir un peque�o o gran n�mero de canciones que desea escuchar en ese momento pero sin crear una playlist permanente.
\end{minipage}
\end{figure}
\pagebreak
Una vez que se ha mostrado espec�ficamente las modificaciones realizadas a las diferentes interfaces del reproductor Mp3, se procede a realizar el sprint de esta semana.
\newpage
\includepdf{Sprint}
\end{document}