\documentclass[12pt,a4paper]{article}
\usepackage[latin1]{inputenc}
\usepackage[spanish]{babel}
\usepackage{amsmath}
\usepackage{amsfonts}
\usepackage{amssymb}
\usepackage{pdfpages}
\usepackage{graphicx}
\usepackage[left=2cm,right=2cm,top=2cm,bottom=2cm]{geometry}
\author{Brito Orlando, Huac�n Linda, Rosales Joshua, Salazar Jes�s}
\begin{document}
\includepdf{caratula}
\title{\textbf{INFORME PRIMER PARCIAL}}
\maketitle
El presente informe fue realizado con la finalidad de dar a conocer los cambios implementados en la aplicaci�n a ra�z de las correcciones realizadas por nuestro cliente (Docente), a m�s de dar a conocer las nuevas implementaciones realizadas al reproductor Mp3 con base en las planificaciones descritas en el backlog del producto, en este caso los backlog's de las diferentes interfaces de pantalla.
\center{\large{\textbf{NUEVAS IMPLEMENTACIONES}}}

\begin{figure}[h!]
\begin{minipage}{0.5\textwidth}
\centering \includegraphics[width=0.75\textwidth]{1.png}     
\end{minipage}
\hfill\begin{minipage}{0.5\textwidth}
\textbf{Pantalla principal:} Se realizaron modificaciones en el fondo de la pantalla de la ventana principal, ahora el color de fondo utilizado en la interfaz es el PMS 282 de Pantone. Adem�s se realizaron modificaciones en el tama�o de los botones de minimizar, cerrar y en el bot�n de ajustes (el cual ahora es mas peque�o).
\end{minipage}
\end{figure}

\begin{figure}[h!]
\begin{minipage}{0.5\textwidth}
\centering \includegraphics[width=0.75\textwidth]{2.png}     
\end{minipage}
\hfill\begin{minipage}{0.5\textwidth}
\textbf{Interfaz de ventana en reproducci�n:} Se realizaron modificaciones  de tama�o en los botones de minimizar, cerrar (que ahora son mas grandes) y se cambi� el la imagen que representaba la opcion de ver letra de m�sica, como podemos darnos cuenta, ahora se muestra un micr�fono. 
\end{minipage}
\end{figure}

\begin{figure}[h!]
\begin{minipage}{0.5\textwidth}
\centering \includegraphics[width=0.75\textwidth]{3.png}      
\end{minipage}
\hfill\begin{minipage}{0.5\textwidth}
\textbf{Vista de ventana en reproducci�n; ver letra de m�sica.-} Al momento de dar clic el ver letra de m�sica se iniciar� por defecto la vista de la letra , en donde el usuario puede leer y  utilizar la barra de desplazamiento ubicada a lado, para poder bajar y subir, observando de �sta manera la letra completa de su canci�n. 
\end{minipage}
\end{figure}

\begin{figure}[h!]
\begin{minipage}{0.5\textwidth}
\centering \includegraphics[width=0.75\textwidth]{4.png}      
\end{minipage}
\hfill\begin{minipage}{0.5\textwidth}
\textbf{Interfaz de �lbumes.-} La ventana de b�squeda por �lbum ser� similar a la de canciones, con la diferencia se van a mostrar los �lbumes existentes, tambi�n contar� con una barra de desplazamiento en caso de que la lista de �lbumes sea larga. La flecha azul sirve para que el usuario pueda regresar a ver los �lbumes de nuevo. 
\end{minipage}
\end{figure}
  
\begin{figure}[h!]
\begin{minipage}{0.5\textwidth}
\centering \includegraphics[width=0.75\textwidth]{5.png}
\end{minipage}
\hfill\begin{minipage}{0.5\textwidth}
\textbf{Interfaz de g�nero.-} La ventana de b�squeda por g�nero mostrar� 2 subventanas, en la primera se mostrar�n los generos existentes y en la segunda estar�n todos los g�neros que hayamos seleccionado para escuchar despu�s. Aqu� tambien fue implementada la flecha azul para volver a ver un g�nero.
\end{minipage}
\end{figure}    
\pagebreak
Una vez que se ha mostrado espec�ficamente las modificaciones realizadas a las diferentes interfaces del reproductor Mp3, se procede a realizar el sprint de esta semana.
\newpage
\includepdf{Sprint}
\end{document}