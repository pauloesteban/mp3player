\documentclass[12pt,a4paper]{article}
\usepackage[latin1]{inputenc}
\usepackage[spanish]{babel}
\usepackage{amsmath}
\usepackage{amsfonts}
\usepackage{amssymb}
\usepackage{graphicx}
\usepackage[left=4cm,right=3cm,top=4cm,bottom=3cm]{geometry}
\author{grupo recordatorio de voz}


\begin{document}
\begin{center}

%%%%%%%%%%%%%%%%%%%%%%%%%%%%%%%%%%%%%%%%%%%%%%%%%%%%
%%%%%%%%%%%%%%%%%%%%%%%%%%%%%%%%%%%%%%%%%%%%%%%%%%%%

\includegraphics[scale=1]{i.png} 


\begin{center}
\vspace{1cm}
\textbf{UNIVERSIDAD T�CNICA ESTATAL DE QUEVEDO} 


\vspace{2ex}
\textbf{FACULTAD DE CIENCIAS DE LA INGEN�ER�A}


\vspace{3ex}
\textbf{ESCUELA DE ELECTRICA}

\vspace{3ex}
\textbf{M�DULO VII}

\vspace{3ex}
\textbf{UNIDAD DE APRENDIZAJE}

\vspace{3ex}
INTERACCI�N HOMBRE MAQUINA

\vspace{3ex}
\textbf{TEMA:}

\vspace{3ex}
\textbf{AGREGAR  LIBRER�AS  }
\vspace{1ex}

\textbf{CARRERA:}

INGENIER�A EN TELEM�TICA
\vspace{1ex}

\textbf{AUTORES:}  
\vspace{1ex}

Orlando Brito  
\vspace{1ex} % dar espacio

  Erick Tirado % Negrilla 
\vspace{1ex} % dar espacio

  Alejandra Flores % Negrilla 
\vspace{1ex} % dar espacio

  Jaime Troya  % Negrilla 
\vspace{1ex} % dar espacio

\textbf{CARRERA:}  

Ingenier�a en telem�tica % Negrilla 
\vspace{2ex}

\textbf{TUTOR:} 

Ing. Chiliguano Torres
\end{center}
\vspace{1ex}


\begin{center}  
\textbf{PERIODO:}

2016 - 2017
\end{center}

%%%%%%%%%%%%%%%%%%%%%%%%%%%%%%%%%%%%%%%%%%%%%%%%%%%%
%%%%%%%%%%%%%%%%%%%%%%%%%%%%%%%%%%%%%%%%%%%%%%%%%%%%
\vspace{1cm}

\textbf{INTRODUCCI�N} 
\end{center}
\vspace{1cm}
\textbf{1. Introducci�n:}

\vspace{1ex}
Requerimos introducir librer�as adicionales para el funcionamiento de la aplicaci�n mp3player con las cuales se da un gran avance a la aplicaci�n  
\vspace{3ex}

\textbf{2. Integrantes :}

Los nuevos miembros de la aplicaci�n mp3player y sus respectivos cargos son:
\vspace{1ex}


\textbf{ Jefe de producto}

	Orlando Brito - Product Owner
	\vspace{1ex}
	
\textbf{ L�der de equipo (o facilitador)}

	Jaime Troya - ScrumMaster
	\vspace{1ex}
	
\textbf{ Colecci�n de programadores}

	Erick Tirado - Scrum Team
	
	Alejandra Flores - Scrum Team
	
	Andr�s Saltos - Scrum Team
	
	\vspace{3ex}
\textbf{Proceso para la instalaci�n de las Librer�as en mp3payer. }
\vspace{3ex}

\textbf{descargar: }
descargamos el contenido que se encuentra en el repositorio mp3player el cual es jlgui3.0.zip 
Link de descarga : https://github.com/pauloesteban/mp3player


\includegraphics[scale=0.5]{1.png} 

\textbf{descomprimir }

damos clic derecho a la archivo jlgui3.0.zip y descomprimimos el contenido en la carpeta peppermusic de java 

\includegraphics[scale=0.5]{2.png} 



\textbf{agregar librer�as }

abrimos el NetBeans y nos vamos a la carpeta PepperMuisc en la cual le vas a dar a + para que se despliegue todo el contenido de la carpeta, nos dirigimos a Libraries en donde daremos clic derecho para desplegar la opciones de las cuales escogeremos Add JAR/Folder          

\includegraphics[scale=0.5]{3.png} 


	\vspace{1ex}

se nos desplegara una ventana en la cual buscaremos donde guardamos la carpeta descomprimida (jlgui3.0.zip), en donde se mostrara la primera librer�a que vamos a ingresar la cual es jlgui3.0.jar en donde le daremos a abrir para poderla ingresar.

\includegraphics[scale=0.5]{4.png} 

\vspace{1ex}
aremos el mismo procedimiento pero ahora vamos a buscara la carpeta "Lib" en la cual abriremos y se mostrara una serie de librer�as que usaremos as� como ingresamos la primera librer�a aremos lo mismo para ingresar todas esa librer�as que son necesarias para el programa mp3player.

\includegraphics[scale=0.5]{5.png} 

\vspace{1ex}
\includegraphics[scale=0.5]{6.png} 

\end{document}